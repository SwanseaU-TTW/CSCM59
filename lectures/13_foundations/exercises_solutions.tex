% Options for packages loaded elsewhere
\PassOptionsToPackage{unicode}{hyperref}
\PassOptionsToPackage{hyphens}{url}
%
\documentclass[
  14pt,
]{extarticle}
\usepackage{lmodern}
\usepackage{amssymb,amsmath}
\usepackage{ifxetex,ifluatex}
\ifnum 0\ifxetex 1\fi\ifluatex 1\fi=0 % if pdftex
  \usepackage[T1]{fontenc}
  \usepackage[utf8]{inputenc}
  \usepackage{textcomp} % provide euro and other symbols
\else % if luatex or xetex
  \usepackage{unicode-math}
  \defaultfontfeatures{Scale=MatchLowercase}
  \defaultfontfeatures[\rmfamily]{Ligatures=TeX,Scale=1}
\fi
% Use upquote if available, for straight quotes in verbatim environments
\IfFileExists{upquote.sty}{\usepackage{upquote}}{}
\IfFileExists{microtype.sty}{% use microtype if available
  \usepackage[]{microtype}
  \UseMicrotypeSet[protrusion]{basicmath} % disable protrusion for tt fonts
}{}
\makeatletter
\@ifundefined{KOMAClassName}{% if non-KOMA class
  \IfFileExists{parskip.sty}{%
    \usepackage{parskip}
  }{% else
    \setlength{\parindent}{0pt}
    \setlength{\parskip}{6pt plus 2pt minus 1pt}}
}{% if KOMA class
  \KOMAoptions{parskip=half}}
\makeatother
\usepackage{xcolor}
\IfFileExists{xurl.sty}{\usepackage{xurl}}{} % add URL line breaks if available
\IfFileExists{bookmark.sty}{\usepackage{bookmark}}{\usepackage{hyperref}}
\hypersetup{
  hidelinks,
  pdfcreator={LaTeX via pandoc}}
\urlstyle{same} % disable monospaced font for URLs
\setlength{\emergencystretch}{3em} % prevent overfull lines
\providecommand{\tightlist}{%
  \setlength{\itemsep}{0pt}\setlength{\parskip}{0pt}}
\setcounter{secnumdepth}{-\maxdimen} % remove section numbering
\pagenumbering{gobble}
\ifluatex
  \usepackage{selnolig}  % disable illegal ligatures
\fi

\author{}
\date{}

\begin{document}

\hypertarget{example-database}{%
\section{Example database}\label{example-database}}

The database has 6 relations. You may assume that the same column name
in 2 tables indicates a primary/foreign key pair:

\begin{itemize}
\tightlist
\item
  \textbf{Branch}(Bname, Bcity, assets)
\item
  \textbf{Customer}(Cname, street, city)
\item
  \textbf{Account}(Anumber, Bname, balance)
\item
  \textbf{Loan}(Lnumber, Bname, amount)
\item
  \textbf{Depositor}(Cname, Anumber)
\item
  \textbf{Borrower}(Cname, Lnumber)
\end{itemize}

\newpage

\hypertarget{relational-algebra}{%
\section{Relational algebra}\label{relational-algebra}}

The 5 operators are:

\begin{itemize}
\tightlist
\item
  Union
\item
  Difference
\item
  Cartesian Product (times, *)
\item
  Selection (where)
\item
  Projection (\{\})
\end{itemize}

\hypertarget{relational-calculus}{%
\section{Relational calculus}\label{relational-calculus}}

\(\{t | P(t)\}\)

\begin{itemize}
\tightlist
\item
  \(t\) is the set of tuples
\item
  \(P(t)\) is a predicate function
\item
  \(t.A\) or \(t[A]\) means attribute \(A\) of tuple \(t\)
\item
  \(t \in r\) means tuple \(t\) in relation \(r\)
\end{itemize}

\hypertarget{useful-first-order-logic-primitives}{%
\subsection{Useful first-order logic
primitives}\label{useful-first-order-logic-primitives}}

\begin{itemize}
\tightlist
\item
  \(\forall\): forall. \(\forall t \in r P(t)\) The condition must be
  true for all tuples.
\item
  \(\exists\): exists. \(\exists t \in r P(t)\) There must be at least
  one tuple where the condition holds true.
\item
  comparisons: \(<\), \(=\), \(>\), \(\leq\), \(\geq\), \(\neq\)
\item
  connectives: not (\(\neg\)), if (\(\Rightarrow\)), iff (if and only
  if, \(\iff\)), and (\(\land\)), or (\(\lor\))
\end{itemize}

\newpage

\hypertarget{example-query-1}{%
\section{Example query 1}\label{example-query-1}}

Find the loan number for each loan of an amount greater than \(1200\)

\hypertarget{relational-algebra-1}{%
\subsection{Relational algebra}\label{relational-algebra-1}}

(Loan WHERE amount\textgreater1200)\{Lnumber\}

\vspace{7cm}

\hypertarget{relational-calculus-1}{%
\subsection{Relational calculus}\label{relational-calculus-1}}

\begin{equation*}
\left\{ t.Lnumber | t.amount > 1200 \right\}
\end{equation*}

\newpage

\hypertarget{example-query-2}{%
\section{Example query 2}\label{example-query-2}}

Find the branch name, branch city, and loan amount for loans of more
than \(1200\)

\hypertarget{relational-algebra-2}{%
\subsection{Relational algebra}\label{relational-algebra-2}}

((Loan WHERE amount\textgreater1200) JOIN Branch)\{Lnumber, Bname,
Bcity\}

\vspace{7cm}

\hypertarget{relational-calculus-2}{%
\subsection{Relational calculus}\label{relational-calculus-2}}

\[
\left\{ t.Lnumber, t.Bname, t.Bcity | \exists l \in \text{Loan}, t.Bname=l.Bname \land l.amount > 1200 \right\}
\]

\newpage

\hypertarget{example-query-3}{%
\section{Example query 3}\label{example-query-3}}

Find the name of all customers with a loan from the Swansea branch and
the city in which they live.

\hypertarget{relational-algebra-3}{%
\subsection{Relational algebra}\label{relational-algebra-3}}

((((Branch WHERE city=Swansea) JOIN Loan JOIN Borrower JOIN Customer)
JOIN (Branch b2 ON b2.Bcity=Customer.city JOIN Loan l2 ON
l2.Bname=b2.Bname AND l2.Lnumber=Borrower.Lnumber)) \{CName\}

\vspace{7cm}

\hypertarget{relational-calculus-3}{%
\subsection{Relational calculus}\label{relational-calculus-3}}


\begin{multline*}
\left\{ c.Cname | \exists b1 \in \text{Branch}, \exists b2 \in \text{Branch}, \exists l1 \in \text{Loan}, \exists l2 \in \text{Loan}, \right. \\
\exists b \in \text{Borrower}, \exists c \in \text{Customer}, \\
b1.city=Swansea \land b1.Bname=l1.Bname \land l1.Lnumber=b.Lnumber \land \\
c.Cname=b.Cname \land c.city=b2.city \land b2.Bname=l2.Bname \land \\
\left. l2.Lnumber=b.Lnumber \right\}
\end{multline*}

\end{document}
